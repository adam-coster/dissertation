\section{The information encoding problem}
\label{introduction:encoding}


An analogous problem to the one we face when trying to understand
cell signaling is the following.
Going back to the example of human speech, how would
alien scientists with no concept of sound think that we
communicate? Perhaps they would observe us over long periods of time, 
and eventually correlate certain patterns
of mouth movements by one human to some behavioral task
performed by another. These aliens
might quite reasonably infer that we encode
communicated messages into mouth movements that are then decoded visually
by the recipient.


Indeed, that putative encoding is a reasonable
approximation of the true encoding, largely because mouth movements
are part of the encoding process that is used by the full sound-based
encoding. This is, of course, why humans can learn to accurately ``read lips.''
Some of these alien scientists would eventually notice that we can
still communicate in the dark, and claim that this discovery shows that
the original encoding model was incorrect. Importantly, breakdown of the model
in this context did not mean that it was incorrect, it simply meant that
some part of the message was being carried through an additional, unobserved channel (sound).


We face the same issue when trying to identify $\vec{S}$ and
$\vec{R}$ for studying cell signaling. Communication via molecules
is so outside of our experience that we have no choice but to make
educated guesses as to how it might work in each context.
There are many approaches for converting biological
data into models, but how do we know what the relevant data
are? In the broad sense, the relevant data are whatever the
cell ``cares about.'' Unfortunately we neither know what
signals a cell is listening to, nor into what form it encodes
this information. If we don't know $\vec{S}$, and
we don't know $\vec{R}$, how can we possibly determine $f$?
I refer to this generally as the ``encoding problem.''


\subsection{Identifying inputs and outputs}


The first difficulty we face is the determination of $\vec{S}$. 
We frequently assume that the property
of signal that the cell cares about is its
concentration (as for a ligand or drug) \cite{Becker2010,Cheong2011}.
From a biochemical perspective this is a sensible guess for what
is being encoded, since we understand biology as a collection of
intermolecular interactions that have binding constants, may show
cooperativity, and that have behaviors that fit onto Hill curves.
This leads to the further expectation that the
input concentration is saturable, following some form of a sigmoid
curve. The assumption that cells sense absolute concentrations need not
necessarily hold true, however, as various pathways can instead show fold-change
detection \cite{Goentoro2009,Shoval2010,Lee2014}. Further, responses need
not follow typical sigmoid curves, as they can sometimes show linear responses 
over a large range of concentrations \cite{Becker2010}.


For $\vec{R}$, we often make a similar assumption that cells encode the
received signal into intracellular concentrations of some factor.
Indeed, this concentration-based encoding defines
morphogenic signaling, which
is typically thought to convert external ligand concentrations into active transcription
factor concentrations. But again the concentration property need not be
the value of $R$ that encodes the sensory input.


For example, the Tumor Necrosis Factor/Nuclear Factor
kappa B (TNF/NF-$\kappa$B) pathway
does show ligand concentration-dependent increases in its nuclear transcription
factor accumulation, but in such a noisy way that single cells may not be able
to accurately sense the absolute ligand concentration \cite{Cheong2011}.
This implies either that single cells are
poor signal processors or that, alternatively, the absolute ligand concentration only
partially encodes the information that the cells are using.
The latter may be the case,
as recent work indicates that the information content of TNF
concentrations is more accurately encoded into the fold-change of transcription
factor activity \cite{Lee2014}.


Another alternative to encoding to or from molecule concentration is the use of
temporal information, such as integration over time or oscillatory behavior
\cite{Tay2010,Kang2011}. Therefore, care should be taken with the assumption
that absolute concentration is of utmost informational value to the cell. This assumption
is difficult to test, however, as there may be a concentration dependence
even if this is not the primary encoding that the cell uses,
as is the case for TNF/NF-$\kappa$B signaling and for the analogy
of mouth movements in human speech.


\subsection{Cellular variability}
\label{introduction:variability}

For a given approximation of $\vec{S}$ and $\vec{R}$,
is it fair to assume that this approximation is equally meaningful
for all cells in the population? Most of our knowledge of signaling stems
from population-based measurements of cellular responses, for example from
Western blots, microarrays, and other common cellular lysate-based
methods. Such methods yield averaged cellular behaviors, therefore
making the implicit assumption that this average reflects individual
cell behavior \cite{Snijder2011,Altschuler2010,Huang2009}.
If this assumption is incorrect, such that our measured values of $R$
do not reflect any real cellular behaviors,
then the properties of $f$ that we infer will be incorrect.


Indeed, many studies have shown that this assumption of cellular homogeneity
is unjustified. Dramatic examples include the classic demonstration
that single \frog\ embryos have switch-like instead of graded
behavior \cite{Ferrell1998}, the finding that various factors thought to be correlated
during adipocyte differentiation were only correlated in a small
subset of cells while being anti-correlated in others \cite{Loo2009}, and the
discovery that
population-averaged measurements were hiding the ultra-sensitivity
of temporally asynchronous bacterial motors \cite{Cluzel2000}.


How can single cells display behaviors different from the population average?
Take the trivial case: a tissue sample may include many different cell types
that have quite different properties. For example, the intestinal
epithelium contains highly-secretory goblet cells and highly-absorptive
enterocytes, but the average behavior of these cells might be neither secretory nor
absorptive. In a less
trivial case, single cells within an apparently homogeneous
cultured cell line can also exhibit cell-to-cell differences,
even if they are derived from the same clone \cite{Singh2010}.
Such within-cell type differences could be due to asynchrony in cell cycle position
\cite{Sigal2006a} or to asynchrony of other phenotypic states
\cite{Kobayashi2009,Kalmar2009,Tay2010,Loo2009}.


Cultured ``homogeneous'' populations can thus show single-cell variation due to an
asynchrony in temporal movement between stable phenotypic types, but they
can also vary more stochastically. Randomness in cellular phenotypes
might stem from asymmetry in inherited properties after stem cell
division, for example \cite{Spencer2009,Navarro2012,Chang2008,Kalmar2009}. 
Additionally, because genes are
typically present in only two copies, and a finite number of molecules
mediate the process of transcription, it is necessarily a noisy
process \cite{Elowitz2002,Munsky2012,Sanchez2013} that can also generate
cellular variability. Note that, without careful temporal studies, variation
due to asynchrony in stable states cannot be easily differentiated from
that due to rapid movement between unstable states.


Outside of fully distinct cell types, why do cells display such variability?
There is no clear answer to this question, though there are many potential
explanations. Perhaps the use of molecules to process information is simply
so inaccurate that it must generate extensive noise. Maybe cells can work
around the noise we that see, so that their behaviors are more precise
than they appear. Or perhaps cells have
adapted to deal with such noise by either suppressing
or making use of it. Indeed, in some cases variability
can be useful to the population \cite{Eldar2010}, as single cells may generate
subpopulations with useful functions. An example is cellular
differentiation in mammals, where stem cells need to choose whether to
differentiate, and to which fate \cite{Chang2008,Kalmar2009,Navarro2012}. In other cases
subpopulations may have resistance to toxic
environmental stresses \cite{Veening2008,Slack2008,Singh2010}, though we
should be cautious with just-so explanations for these kinds of links.


\subsection{Context-dependency}
\label{introduction:encoding:context}

Transcriptional
networks and chromatin state are highly cell type- and
environment-dependent. As a consequence, properties that are essential
to signaling may vary between experimental systems. Examples include
concentrations of cellular receptors, pathway
modulators, and effectors. Such
differences in cellular properties and in the microenvironments
in which cells live are often collapsed into the term ``cellular context.''


Cellular context also includes properties
of other signaling pathways, which is important because
canonical pathways may be not be isolated information channels.
Thus, knowing the likely extent of crosstalk
is important, though general pathway interconnectedness is
difficult to measure. Some types of signaling are particularly interconnected,
such as for the growth factors that modulate
downstream kinase cascades due to the use of higly overlapping downstream components
\cite{Janes2006}. On the other hand, other pathways
may be much less interconnected, as I show in this dissertation
for several key morphogenic signals \arp{insulation:introduction}.


Attempts to quantify pathway interconnectedness are few and are
necessarily limited by what can be practically measured, along with the issues I
have already noted in this section.
Computational work seems to show that signaling through one
pathway can be broadly modulated by properties of the entire
cell signaling network \cite{Domedel-Puig2011}, while experimental
work shows that non-additive inter-pathway crosstalk is a sparse phenomenon
even for high-order combinations
of up to five signaling pathways \cite{Natarajan2006,Hsueh2009}.

