\section{Dissertation aims}
\label{introduction:aims}

This chapter provided an abstract foundation on the
problems faced in the study of cellular signaling. In particular,
I focused on our lack of clear knowledge about how cells encode signals into
intracellular models, and how this lack of clarity may
be leading us to unnecessarily complex signaling pathways.
In this dissertation I present a case study of one
such apparently-complex signaling phenomenon, that of
cross-pathway integration between Wnt and Transforming
Growth Factor Beta signaling, wherein I demonstrate that the
interactions are simpler than is currently believed.


In \ar{pathways:introduction} I review the literature on the 
classic developmental signaling
pathways that are the focus of my case study, and the
claimed mechanisms off crosstalk between them.
These are the Wnt and
Transforming Growth Factor Beta pathways. I chose these signaling
networks because they are highly studied, and so have
well-established approximations of what cells care about both for inputs
$\vec{S}$ and outputs $\vec{R}$. Further, both Wnt and TGFB have
relatively clean canonical forms that do not share any core
components, and yet there is a large body of work that
ties these pathways together.


In \ar{imaging:introduction} I establish rigorous, quantitative methods for fluorescence
microscopy image analysis that I use to study crosstalk
between the Wnt and TGFB pathways. The study of crosstalk requires
a large number of experimental conditions, and the literature on 
crosstalk generally lacks single-cell resolution. I therefore chose high-throughput
immunofluorescence microscopy as my primary experimental method.
Precise single-cell measurements
are essential to quantifying single-cell
phenotypes, and so I focus on the discussion on how experimental
error can be removed or measured.


In \ar{insulation:introduction} I make use of the conceptual approach to
cellular signaling described in \ar{introduction:introduction},
the body of literature about the Wnt and TGFB signaling pathways
reviewed in \ar{pathways:introduction}, and the quantitative methodologies
established in \ar{imaging:introduction}, to experimentally determine
the degree of Wnt/TGFB crosstalk during signal transduction. There, I
demonstrate the finding that these pathways are in fact insulated from
one another, thus making a case for simplicity in morphogenic signal
integration.


\subsubsection{Reading this dissertation}


Each chapter is relatively independent, though
the reader may find some points confusing without reading earlier
chapters. \ar{imaging:introduction}, on quantitative single-cell imaging,
in particular can
stand alone and so I have placed it near the end so as to not disrupt
the flow of the biological content of Chapters \ref{pathways:introduction}
and \ref{insulation:introduction}.
The imaging chapter should be a useful guide to any biologist or analyst
in need of a conceptual and practical reference for rigorous image analysis.
Those readers primarily interested in the biology of Wnt and TGFB signaling
crosstalk can skip \ar{imaging:introduction} without significant loss of coherence.

