\section{Methods}
\label{insulation:methods}


The rationale and general methodology for image correction
and analysis are provided in \ar{imaging:introduction}.
Specific experimental details are provided in the image captions;
this section provides information that is
broadly applicable or more detailed than is appropriate for a figure legend.


\textbf{Cell culture.} 
I maintained all cells in
RPMI1640 (Cellgro \#10-040) with 5\% FBS
(Gemini Bio-Products \#100-106) with
antibiotics/antimycotics under standard tissue
culture conditions (37$^{\circ}$C, 5\% CO2). In preparation
for imaging, I plate $\sim 10^4$ cells per well
of a 96-well plastic plate (Corning \#353219) and a fifth
of that for 384-well glass plates (Nunc \#164586). I use
DMEM (Gibco \#11965-126) at the time of seeding when starvation
conditions are needed. In either case,
cells are left to adhere overnight before being treated the following day.
For treatments, I dilute recombinant protein into the same
media that the cells are grown in. The proteins used for treatment are
listed in \ar{table:insulation:proteins}.
Most of the plates used in this chapter are 384-well glass plates.
Human colonic epithelial cells were a kind gift from
Dr. Jerry Shay (University of Texas Southwestern, Dallas TX)
\cite{Roig2010}. The other cell lines were obtained from 
American Type Culture Collection (Manassas, VA).

   \begin{table}[!bt]
    \centering
	\footnotesize
    \caption[List of recombinant proteins.]
    { Recombinant proteins used in this chapter. 
      Vendors: CST, Cell Signaling
      Technology Inc (Danvers, MA); 
      Life, Life Technologies (Grand Island, NY);
      R\&D, R\&D Biosystems (Minneapolis, MN).
    }
    \label{table:insulation:proteins}
    \begin{tabular}{llll}
    \hline
    Protein   & Vendor & Catalog \# & Lot \# \\ \hline
    BMP4        & CST  & 4697    & \\
    DKK1        & R\&D & 5439-DK & SMR2713041\\
    Noggin      & R\&D & 6057-NG & \\
    \tgf 1      & Life & PHG9204 & \\
    \tgf 3      & CST  & 8425    & \\
    Wnt3A-LP    & R\&D & 5036-WN & RSK31 \\
    Wnt3A-HP/CF & R\&D & 5036-WNP/CF & SVH0813081 \\
    Wnt5A-LP    & R\&D & 645-WN    & \\
    Wnt5A-LP/CF & R\&D & 645-WN/CF & MCR4513111 \\
    \hline

    \hline
    \end{tabular}
    \end{table}


\textbf{Gene expression.}
I plated cells in 96-well plastic plates as above and left them to
adhere overnight before treatment the following day.
Each treatment was performed in triplicate wells.
I used an Ambion Cells-to-CT
kit (Life Technologies) to extract mRNA according to manufacturer instructions,
and passed these samples on to the UT Southwestern
Medical Center Microarray core facility for
cDNA library preparation and TaqMan qPCR. The samples were
given obfuscating identifiers and within-plate positions
so as to blind the core facility
to the treatments and replicates.
TaqMan probes (Applied Biosystems) comprised
Smad7 (Hs00998193\_m1), Axin2 (Hs00610344\_m1),
and 18S rRNA (Hs99999901\_s1).
The core facility returned the raw threshold cycle ($C_t$)
values for each probe/sample combination, which I first internally
normalized by 18S rRNA levels to yield
$C_{t(probe,norm)}=C_{t(probe)}-C_{t(18S)}$ and then 
converted these values to fold-change over control.


\textbf{RNA interference.}
I used a Dharmacon Smad4 siGENOME SMARTpool
(GE Life Sciences, \#M-003902-01) to ablate Smad4 in HCEC cells.
For transfection I used Lipofectamine RNAiMax
(Life Technologies) according to manufacturer instructions.
The transfection media was left on cells for 48hrs, after
which I replaced the media with fresh media for >2hrs prior
to experimental treatment.


\textbf{Western blots.}
SDS-PAGE and western blotting were performed using standard techniques.
I plated cells in 6-well dishes and treated them the following day.
After treatment, I washed the wells with ice-cold PBS
and then lysed with RIPA buffer (50 mM Tris (pH 8.0), 150 mM NaCl,
1\% NP40, 0.5\% deoxycholic acid, 0.1\% SDS, 0.5 mM EDTA)
containing protease and phosphatase inhibitors (Sigma Aldrich).
Antibodies used are shown in \ar{table:insulation:antibodies},
generally with 1:1000 dilutions
(the CST rabbit Smad4 antibody was used
for blotting).


\textbf{Immunostaining}.
All solutions are made in PBS
(Gibco \#70013). All wash steps are repeated 3 times with 0.1\%
Tween20 (Fisher \#BP337). Antibodies are diluted into 2.5\%
BSA (Fisher \#NC9871802). Cells are fixed in 4\%
paraformaldehyde (Electron Microscopy Sciences \#15710) for
10min, permeabilized with 0.2\% Triton X-100 (Sigma \#93443)
for 10min, then washed. I then incubate the samples overnight
at 4$^{\circ}$C with appropriate primary antibodies
\arp{table:insulation:antibodies}.
After washing, I secondary-stain for 2hrs at
room temperature with
added 2.5\textmu{g}/mL Hoechst. 
Finally, I wash the samples once more before
storing them in PBST for imaging.
For all experiments, I prepare
6-18 uniformly-fluorescing wells for measuring image
shading, using the same dissolved
secondary antibodies and Hoechst.


    \begin{table}[!bt]
    \centering
	\footnotesize
    \caption[List of antibodies.]
    { Antibodies used in this dissertation. Vendors: CST, Cell Signaling
      Technology Inc (Danvers, MA); BD, BD Biosciences (San Jose, CA);
      Abcam (Cambridge, MA); Life, Life Technologies (Grand Island, NY);
      R\&D, R\&D Biosystems (Minneapolis, MN).
      The indicated dilutions are for immunofluorescence unless appended
      with a `w' for Westerns. The appendix `b' indicates use as a blocking antibody.
    }
    \label{table:insulation:antibodies}
    \begin{tabular}{llllll}
    \hline
    Target     & Dilution & Source & Vendor & Catalog \# & Lot \# \\
    \hline
    Smad2/3     &  1/1000  & Rabbit &   CST & 8685   & 3         \\
    pSmad1/5/8  &  1/100   & Rabbit &   CST & 9511   & 8         \\
    \bcat\      &  1/100   & Mouse  &   BD  & 624084 &           \\
    Smad4       &  1/100   & Rabbit &   CST & 9515   & 4         \\
    Smad4       &  1/100   & Mouse  & Abcam & 3219   & GR94411-1 \\
    Smad1       &  1/100   & Rabbit &   CST & 6944   & 2         \\
    Smad2       &  1/100   & Rabbit &   CST & 5339   & 4         \\
    pSmad2/3    &  1/100   & Rabbit &   CST & 8828   &           \\
    H3B         &  1/1000w & Rabbit &   CST & 9715   &           \\
    \tgf 1-3    &  1/1000bw& Mouse  &  R\&D & MAB1835   & CCI1512031  \\
    AlexaFluor 546 Anti-Mouse IgG (H+L) &  1/1000 & Goat &  Life & A11003 & 1256168   \\
    AlexaFluor 488 Anti-Rabbit IgG (H+L) &  1/1000 & Goat &  Life & A11008 & 1470706   \\
    \hline
    \end{tabular}
    \end{table}


\textbf{Imaging.} I imaged all stained plates on a
Nikon Eclipse Ti-E2000 microscopes controlled by NIS
Elements version 4, with an Andor Zyla sCMOS 11-bit camera. I wrote custom
image coordinate-generating software in Python for increased
stage precision. To obtain the signal from the camera alone, I take
detector images at low exposure times with no light source.


\textbf{Image Correction.}
The background and shading correction
followed the pipeline described in \ar{correction:pipeline}
using custom Matlab software.


\textbf{Segmentation and quality control.} For
all analyses, I wrote a custom Matlab threshold segmentation algorithm 
to automate detection of nuclei using the
Hoechst fluorescence channel. 
I manually set filters for size and shape, by cell type, to remove objects that are
likely to be artifacts. For each nucleus, I then extract
its area (in pixels) and the total of all pixel intensities
for all imaged channels. Finally, I follow the DNA-based
quality control and single-cell regression-based correction
described in \ar{imaging:variation} using custom R software.
The resulting high-quality G1-phase cells are used for analysis.


\textbf{Mutual information measurement.} I implemented
the unbiased mutual information algorithm as described in
\cite{Cheong2011} and discussed in \ar{imaging:information},
using a custom Matlab script. 
