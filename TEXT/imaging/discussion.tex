\section{Discussion}
\label{imaging:discussion}

Unlike other quantitative single-cell
methods, such as flow cytometry, the subcellular resolution of
image data allows for a stupefyingly
large number of feature measurements, and there are no standardized
practices for choosing or implementing these
features. Further, identification of individual
cells takes place at the level of software, not
hardware. Quantitative imaging is thus an
exceedingly difficult task outside of the labs
that specialize in it, and no two of these labs
are likely to converge on the exact same solutions.


With imaging we can directly see the beauty of the
biology we are studying, and the high information
content of images makes this type of data boundless
in its potential utility. We are currently not
meeting this potential, however, and I firmly believe
that this is due to an absence of established standards
and approaches that would make quantitative imaging more broadly accessible
and interpretable. To that end, I hope that this chapter
provides some intuition to those scientists who have
not had the opportunity to work and think extensively
about image data. Further, I hope that my approach
to single-cell image analysis, described in this chapter
and demonstrated in \ar{insulation:introduction} for
a specific biological study, provide useful demonstrations of the utility
of quantitative single-cell analysis.


