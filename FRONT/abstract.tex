% abstract (<=350words, no abbreviations)
% 	Most abstracts contain: 
%	1) Statement of the Problem 
%	2) Procedure or Methods 
%	3) Results 
%	4) Conclusions

\begin{center}

\vspace*{118pt}
\textsc{\myTitle}

\vspace*{66pt}
Adam D. Coster, Ph.D.\\

The University of Texas Southwestern Medical Center at Dallas, 2014\\

\vspace*{33pt}
Lani F. Wu, Ph.D.\\
Steven J. Altschuler, Ph.D.\\

\vspace*{33pt}
\begin{quotation}
How cells integrate external cues in order to make behavioral decisions is a central problem of cell biology. In development and in tissue-homeostasis, cell-fate decisions are made by the integration of multiple morphogenic signals, but how cells convert such combinations of signals into distinct behaviors is not well understood. A major complication is our incomplete knowledge of which signal properties encode the information that cells use for decision-making. A further complication is that the static networks we use to describe cellular signaling pathways are likely to be overly-complex; the true signaling network, in a given cellular context and at a particular point in time, may be much simpler. Using a rigorous and quantitative single-cell imaging approach, I find that such simplicity is present in the integration between Wnt and Transforming Growth Factor Beta (TGFB), which are key developmental pathways. Surprisingly, this insulation extends to the integration of signals within the TGFB superfamily, which are expected to compete for shared components and so interfere with one another during signal transduction. My results thus add clarity to and simplify our understanding of how cells integrate information from the Wnt and TGFB pathways, and further suggest that insulation of signal transduction may be a common feature of morphogenic pathways.
\end{quotation}

\end{center}