\chapter{Conclusion}
\label{conclusion:introduction}


The study of cellular signaling is a difficult one, in large
part because we know so little, \i{a priori}, about what aspects
of a signal a cell cares about and into what intracellular properties
it encodes those signals. It may be, however, that the complexity of
our understanding of signaling is hiding a reality of underlying simplicity
(\ar{introduction:introduction}). In particular, the static
network models that we rely on include links derived from
different timescales (such that activity along one link
should be considered constant relative to activity along another)
as well as links that may only be present in a
subset of true signaling network instances
(e.g. in certain cell types or experimental conditions).


Wnt/\bcat\ and \tgfbsf\ together
provide a case study for this problem of signaling
complexity (\ar{pathways:introduction}).
For each pathway alone there is a lack
of clarity due to idiosyncratic outcomes between
labs and experiments. There is even less
clarity when considering how the two pathways are integrated
by cells in order for them to make decisions.
Much of the confusion
regarding crosstalk between these pathways may stem
from the reliance of published work on overexpression assays,
which can push cells into abnormal states. Further,
the same studies rely on transcription-based
readouts that are measured long after initial pathway 
activation, allowing for transcriptional feedback to confound
the results \arp{pathways:discussion}. 


Therefore, I tested Wnt/\bcat\
and \tgfbsf\ for crosstalk using an assay that relies on
endogenous levels of proteins (to prevent the signaling networks
from being pushed into abnormal states), short timecourses
(to prevent transcriptional feedback from confounding 
interpretation), and direct readouts of signal transduction
\arp{insulation:introduction}.
This study design allowed me to directly test the affect
of signaling through one pathway on signaling through the other.
In this way, I discovered that Wnt/\bcat\ and \tgfbsf\ are in fact
completely insulated from one another during signal transduction.
Further, I found no evidence for the expected intra-\tgfbsf\ inhibition
nor the competition for the shared component, Smad4, thought to cause it.


Though I observed that signaling insulation is a general
phenomenon, I did uncover an instance of context-dependent
transcriptional crosstalk. Importantly, the resulting transcriptional
feedback led to biasing of signaling
activity over longer timecourses. In effect, this is a direct
example of the fact that some network links may exist only under certain
conditions, and that experimental separation of signaling and transcription is essential
to understanding the nature of morphogenic pathway integration.
From these results, I conclude that there exists a surprising simplicity
of interactions between morphogenic pathways: there is a high degree
of insulation between Wnt/\bcat\ and \tgfbsf, allowing nuclei to make
transcriptional decisions based on more-complete models of their environments
(\ar{insulation:discussion}).


For the experimental approach that led to my discoveries,
I relied heavily on quantitative, single-cell fluorescence
microscopy (\ar{imaging:introduction}).
Single-cell imaging is an increasingly available
and invaluable approach
to studying cell biology (and cell signaling in particular).
Unfortunately, single-cell image analysis is a difficult and unsolved problem.
An oft-neglected issue is that of image correction, prior to
analysis. Indeed, the most commonly-used methods throughout the literature
tend to provide incomplete or error-prone image correction.
I discovered a solution to this problem, in the particular case
of high-throughput microscopy, which takes advantage of predictable optical
properties within microtiter plates (\ar{imaging:correction}).


\subsubsection{On negative results}


When I first found that Wnt3A and \tgf 3 are insulated
from one another during signaling, I was hugely disappointed. This disappointment
came, in large part, from having spent a year under the impression
that there actually was crosstalk. I had performed many experiments
and narrowed down the point of inter-pathway crosstalk to the receptor
or ligand level, before I finally discovered contamination of the purified
recombinant Wnt3A ligand (\ar{fig:insulation:contamination}).


The artifactual result itself was not the most disappointing part;
it was instead the feeling that I now had a ``negative result'' and would
therefore have to drop the project completely and start over.
This attitude had been ingrained within me throughout my science
education: I was taught that scientific discovery is about finding
new and exciting things. Or, in any event, that my discoveries would need to be
new and exciting in order to get published and move my career forward.


While scrambling to make the best out of the situation, I slowly
realized that there was not truly a problem. The fact that the crosstalk
I observed was an artifact was a real result: numerous studies
show signaling interactions between these pathways, and the contaminated
reagent is among the most commonly used for studying Wnt/\bcat\
signaling. As long as I
could unambiguously demonstrate the absence of crosstalk, then my result
was not a negative one at all and could potentially reduce some confusion
in the field. One person's negative result is another person's positive
result, as it were.


In any case, there is value to negative results. In recent years there
has been increasing hubbub in the news and in opinion pieces regarding
the state of biomedical research. Of particular concern has been the
irreproducibility of published results and the toxic effects of
using ``impact'' as the primary metric for determining the merit
of both scientific discoveries and of the scientists who make those discoveries
\cite{Schatz2014,Alberts2014}. Why does the emphasis on impact lead to
irreproducibility? 
A needed statistic when evaluating a claim
is the \i{a priori} likelihood of that claim being true;
when the literature is biased towards exciting (i.e. unlikely)
results because scientists do not publish ``negative'' results, 
we cannot accurately
estimate the likelihood of truth for new claims. This problem
is (in)famously addressed in a paper by John Ioannidis
that should be required reading for all biologists \cite{Ioannidis2005}.


\subsubsection{Parting thoughts}


The overarching field of cellular signaling is an exciting one;
understanding how cells communicate and process information
is absolutely fundamental to our understanding of all of
cell biology. Importantly, a deep understanding of this
topic will allow us to more ably manipulate biological systems,
both in medicine and in bioengineering.
To develop that deep understanding, however, it is essential
that we revisit the complex static maps of established cellular signaling
pathways, and then pay careful attention to how time and context
reshape these networks. By identifying underlying simplicity
in cell signaling, we will be better equipped to both understand
and control cellular information processing and decision-making.
