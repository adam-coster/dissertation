\section{Methods}
\label{pathways:methods}


    \begin{table}[!bt]
    \centering
	\footnotesize
    \caption[List of crystal structures used in phylogenetic trees.]
    { Sources of structures used in
    alignments for \autoref{fig:tgfb:trees} and
	\autoref{fig:wnt:trees}. }
    \label{table:pathways:methods:ligandStruct}
    \begin{tabular}{lllc}
    \hline
    Symbol  & Species & PBD ID & Source \\
    \hline
    TGFB1 &  \textit{Homo sapiens}   & 3KFD.A  & \cite{Radaev2010}  \\
    TGFB2 &  \textit{Homo sapiens}   & 2TGI.A  & \cite{Daopin1992}  \\
    TGFB3 &  \textit{Homo sapiens}   & 2PJY.A  & \cite{Groppe2008}  \\
    BMP2  &  \textit{Homo sapiens}   & 1REW.A  & \cite{Keller2004}  \\
	SMAD3 &  \textit{Homo sapiens}   & 1U7F.A  & \cite{Chacko2004}  \\
	SMAD4 &  \textit{Homo sapiens}   & 1U7F.B  & \cite{Chacko2004}  \\
	SMAD7 &  \textit{Homo sapiens}   & 3KMP.A  & \cite{BabuRajendran2010}  \\
	Wnt8  &  \textit{Xenopus laevis} & 4F0A.B  & \cite{Janda2012} \\
	Fz8   &  \textit{Xenopus laevis} & 4F0A.A  & \cite{Janda2012} \\
    \hline
    \end{tabular}
    \end{table}
    

\textbf{Sequence alignments.}
I obtained sequence data from the National Center for Bioinformatics (NCBI)
servers, choosing each time
the top listed isoform from the Genes database
Protein structures are from the
Research Collaboratory for Structural Bioinformatics (RCSB)
protein database (PDB) \arp{table:pathways:methods:ligandStruct}.
I used subsets of these sequences and structures for alignment in
Promals3D \cite{Pei2008} to obtain phylogenetic trees.
This multi-sequence alignment algorithm takes advantage of
structural information for improved alignments, however the
distances in the phylogenies should be considered approximate since the method
for calculating genetic distance is more naive than the alignment method.
The crystal structures were included in the Wnt and Frizzled trees
because of species differences from the primary sequence data.
Structures for the \tgfbsf\ and Smad trees were dropped because
their sequences identically match a subset of the primary sequences.
The trees were drawn with the R package `ape' \cite{Paradis2004}.


For predicting \textit{Trichoplax adhaerens} orthologs of
genes, I used PSI-BLAST \cite{Altschul1997} with some combination of
human and \i{Drosophila melanogaster} reference proteins. I ran one or more
iterations of the algorithm on the seed sequences, and then chose the few top hits
that were dramatically better matches by both query coverage
and identity. Each putative \ta\ ortholog is listed in the appropriate table in
this chapter. These orthologs are in agreement with the literature
\cite{Srivastava2008}.


Specifically, to identify \tgf/BMP
orthologs I used the \tgf 1-3, BMP2/4, and DPP sequences as seeds,
resulting in 3 putative orthologs (>50\% coverage). For the TGFB receptors, I used
the Type I and Type II receptors separately, but each yielded
a large number (>200) of putative matches that had reasonable identity
(>30\%) but low coverage (<50\%) or vice versa. I therefore did not
include these in the phylogenetic tree. For the Smads,
I used the eight human genes and the \fly\ MAD as seeds, yielding three good hits
(>90\% coverage, >30\% identity). For the Wnts, I used all 19 human
genes and fly WG, yielding 2 hits ($\geq$68\% coverage, >30\% identity).
For the Frizzleds, I used all 10 human
genes and fly FZ, yielding 2 hits (>50\% coverage, $\geq$30\% identity).
For \bcat\ I used human CTNNB1 with Junctional Plakoglobin (JUP, also known
as gamma-catenin) and fly arm,
yielding 1 hit (>70\% coverage, >70\% identity).




