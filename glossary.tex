\newglossaryentry{xtalk}
{
  name=crosstalk,
  description={ A transfer of information between signaling components
				of canonically distinct pathways.}
}

\newglossaryentry{encoding}
{
  name=encoding,
  description={ The manner in which information is converted from one type
                to another. For example a piece of text can be encoded into
                a binary format, or the extracellular concentration of a ligand
                can be encoded into the nuclear concentration of a transcription
                factor.}
}

\newglossaryentry{endogenous}
{
  name=endogenous,
  description={ Usually used to refer to the normal cellular
                concentrations of some factor (contrast to overexpression
                and exogenous).}
}


\newglossaryentry{exogenous}
{
  name=exogenous,
  description={ Usually used to refer to an addition to the normal
                concentrations of some factor (contrast to endogenous).
                For example, an added purified ligand is an exogenous
                source of that ligand, and an overexpressed protein
                creates an exogenous pool in addition to the endogenous
                pool.}
}


\newglossaryentry{overexpression}
{
  name=overexpression,
  description={ The exogenous expression of some protein, for example
                by transient expression from a plasmid, constitutive
                expression by a genomically-integrated viral construct,
                or controllable expression from an inducible promoter.
                Generally implies that there is more than a normal quantity of the expressed
                protein.}
}



\newglossaryentry{dna}
{
  name=DNA,
  description={ Deoxyribonucleic acid (only a massochist would use this
                acronym for something else). Used also when referring
                to the total fluorescent Hoechst signal from stained cells, as 
                this molecule intercalates
                into the DNA backbone and thus serves as a proxy for DNA content.}
}

\newglossaryentry{rna}
{
  name=RNA,
  description={ Ribonucleic acid. Used with various prefixes to indicate
                the specific type of this molecule. Types include messenger RNA,
                small interfering RNA, and ribosomal RNA.}
}

\newglossaryentry{uenvironment}
{
	name=microenvironment,
	description={ A term that is thrown around in the literature extensively
				  but frequently (and perhaps purposely) left undefined. Here,
				  it is used to refer to the collection of environmental parameters that
				  a single cell is exposed to. A subset of such parameters include
				  juxtacrine and paracrine signals from neighboring cells, as well
				  as any more global properties (e.g. temperature). In the context of an
				  experiment, this also includes any perturbations that a cell should
				  be able to sense.}
}

\newglossaryentry{canon}
{
  name=canonical pathway,
  description={ A well-established series of biochemical signaling steps,
				which effectively pass information from one molecule to another.
				Often defined by genetic means with epistasis mapping. Also used
				to refer to pathways that are more easily studied than alternatives
                (e.g. compare
				canonical and non-canonical Wnt signaling).}
}

\newglossaryentry{canonWnt}
{
	name=canonical Wnt signaling,
	description={ The branch of Wnt signaling that results in increased
				  intracellular \bcat\ levels (therefore also called
                  Wnt/\bcat\ signaling. This form of Wnt signaling
				  is much easier to study than the others, because \bcat\
				  is easy to measure and is relatively insulated from
				  other signaling pathways.}
}

\newglossaryentry{nonCanonWnt}
{
	name=non-canonical Wnt signaling,
	description={ The branches of Wnt signaling that do not result in increased
				  intracellular \bcat\ levels (in the longer-term, these pathways
				  may inhibit canonical Wnt signaling). This form of Wnt signaling
				  has been difficult to study, because its readouts (including
				  transient Ca$^{2+}$ signaling) are hard to measure and are integrated
				  with many other signaling pathways. For these reasons, it is not
				  clear how many non-canonical pathways there are, how distinct one
				  is from another, and what the impacts of this form of signaling are
				  on canonical signaling (besides general long-term inhibition).}
}

\newglossaryentry{edge}
{
	name=edge,
	description={ A link between nodes (borrowed from graph theory). For a signaling network,
				  indicates some interaction (e.g. \pn) or transfer of information between
				  nodes.}
}

\newglossaryentry{node}
{
	name=node,
	description={ A conceptual unit that may interact with another unit
                 (borrowed from graph theory).
				  For a signaling network, may indicate a protein or protein state.}
}

\newglossaryentry{knockout}
{
	name=knockout,
	description={ The removal of a gene from the genome. Typically used to refer to the removal
				  of both alleles in a diploid organism, though the term ``homozygous knockout'' means
				  this more specifically. Thus a ``Wnt5A knockout mouse'' likely lacks both endogenous
				  alleles of Wnt5A.}
}

\newglossaryentry{frog}{
	name=\textit{Xenopus laevis},
	description={ A frog used as a model organism. Wnt and BMP have been
				  heavily studied in this organism, particularly with respect to development.},
	plural=\textit{Xenopus}
}

\newglossaryentry{fly}{
	name=\textit{Drosophila melanogaster},
	description={ The classic fruit fly model system. If you are a biologist,
				  you cannot be forgiven for having to look up this term. },
	plural=\textit{Drosophila}
}

\newglossaryentry{human}{
	name=\textit{Homo sapiens},
	description={ If you are reading this, you are probably one of these. }
}

\newglossaryentry{ta}{
	name=\textit{Trichoplax adhaerens},
	description={ The most basal known metazoan, with a simple multicellular structure.},
    plural=\textit{T. adhaerens}
}

\newglossaryentry{homology}{
    name=homology,
    description={ Having a shared evolutionary ancestor. Thus genes with
                  high homology have similar sequences and recent ancestry
                  (or high selective pressure). It is important to note
                  that sequence similarity does not necessarily imply
                  homology (i.e. ``homologous'' does not
                  mean ``similar''). The noun form of this term is ``homolog,''
                  which is a more general term than are paralog and ortholog.
                }
}

\newglossaryentry{signaling}{
    name=signaling,
    description={ Also referred to throughout the text as ``cellular signaling''
                  and ``signal transduction.'' I use this term specifically to
                  refer to the process by which an external stimulus is encoded
                  into an internal representation of that stimulus.
                }
}


\newglossaryentry{decision-making}{
    name=decision-making,
    description={ The mapping of an internal model of a signaling event to some
                  response. An example internal model might be the nuclear concentration
                  of a transcription factor, while the decision is then how much of
                  some transcriptional target to produce.
                }
}

\newglossaryentry{paralog}{
    name=paralog,
    description={ Homologous sequences, within the same genome, that resulted from a gene
                  duplication event.
                }
}

\newglossaryentry{ortholog}{
    name=ortholog,
    description={ Homologous sequences, in two different organisms, that resulted from a
                  speciation event (i.e. they are the ``same'' sequence).
                }
}

\newglossaryentry{ligand}{
    name=ligand,
    description={ A protein or other molecule that is recognized by
                  a cellular receptor, thus leading to some internal
                  representation of properties that molecule.
                }
}

\newglossaryentry{transcription factor}{
    name=transcription factor,
    description={ A molecule (typically a protein) that directly binds to DNA,
				  or to other molecules that bind DNA, and thus can cause a change in the transcription
				  rate of a gene.
                }
}

\newglossaryentry{LiCl}{
    name=LiCl,
    description={ Lithium Chloride. Can be used to inhibit \gsk.
                }
}

\newglossaryentry{proteosome}{
    name=proteosome,
    description={ A large protein complex that degrades proteins
                  in a highly regulated manner, typically after those
                  proteins have been modified by the covalent addition of ubiquitin.
                }
}

\newglossaryentry{probe}{
    name=probe,
    description={ Short for ``fluorescent probe,'' used to refer to a
                  fluorescent small molecule or antibody that binds to a specific
                  molecular target.
                }
}

\newglossaryentry{image correction}{
    name=image correction,
    description={ The removal of detector, background, and
                  shading components from an image.
                }
}

\newglossaryentry{immunostain}{
    name=immunostain,
    description={ The use of an antibody to attach a fluorophore,
                  or other measurable item, to a target molecule. 
                }
}

\newglossaryentry{channel}{
    name=channel,
    description={ In the imaging sense, used to refer to a
                  fluorescence color channel (e.g. Hoechst
                  and fluorescein fluoresce in different
                  channels). In the information-carrying sense,
                  a channel is a distinct path of information flow.
                }
}

\newglossaryentry{feature}{
    name=feature,
    description={ A single type of measurement in image
                  analysis. Example features include
                  nuclear area or total cytosolic intensity.
                }
}

\newglossaryentry{paneth cell}{
    name=paneth cell,
    description={ A long-lived cell type living in the base of crypts.
                  It is thought to provide the stem cell niche in the small
                  intestine.
                }
}

\newglossaryentry{detector}{
    name=detector,
    description={ The camera used to acquire fluorescence images.
                  It will typically have a constant baseline value
                  added to images that must be subtracted during image correction.
                }
}

