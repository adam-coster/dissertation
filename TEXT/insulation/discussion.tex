\section{Discussion}
\label{insulation:discussion}


In this chapter I measured the extent of crosstalk
between key morphogenic pathways, using a mechanism-independent
and endogenous approach that kept signaling crosstalk separable
from transcriptional crosstalk.
By doing so, I discovered that canonical Wnt and \tgf\
do not crosstalk at all during signal transduction, and that
\tgf\ and BMP do not compete for Smad4 during signaling.
Both of these results yield important simplifications to
the ever-increasing apparent complexity of signal transduction
that stems from discoveries of putative nodes of crosstalk.
Below, I discuss specific implications in more detail.


\subsection{Morphogenic signaling is insulated}


As I explain in \ar{pathways:discussion}, crosstalk during
signal transduction is likely to decrease the accuracy of the
intra-nuclear model of the extracellular environment. In such
a case, the information that eventually passes to the nucleus
for transcriptional processing represents a simplified view
of the cellular microenvironment, such that the nucleus then
has to make a decision with more uncertainty about the state of the outside world.
By instead allowing information to pass relatively unfiltered
from the extracellular milieu to the nucleus (e.g. by insulating
information channels from one another) the nucleus has access to a more
accurate model of the microenvironment and, presumably,
can then make more useful decisions.


The results presented in this chapter suggest that
the morphogenic Wnt and \tgfbsf\ pathways are indeed isolated
from one another, such that they can each pass information to
the nucleus without cross-pathway interference. As each pathway is only
capable of sending a small amount of information regarding
ligand concentration (not shown, but one can infer this from the broad distributions in Figs.
\ref{fig:insulation:readoutInformation} \&
\ref{fig:insulation:ligandInformation}), maintaining this
information from distinct pathways may be the only way
that a cell can obtain an accurate internal model of its
environment. Therefore, I propose that signaling insulation
may be a general property of morphogenic pathways whose
key outputs are changes to the transcriptional network.


Signaling insulation allows for the internalization of a
somewhat unfiltered view of the cellular microenvironment
(\ar{fig:insulation:wntTgfbSummary}a, cartoon).
Context-dependent transcriptional feedback
(\ar{fig:insulation:wntTgfbSummary}b, cartoon) can then modulate
the signaling pathways over time (either by auto-regulation or
cross-pathway regulation)
(\ar{fig:insulation:wntTgfbSummary}c, cartoon). The result of this nuclear
decision-making would be to essentially bias the
cell's view of what could even be an unchanging environment. Thus,
by mixing insulated signaling with long-term feedback,
cells can maintain a relatively complete internal model
of the environment but interpret that environment in a
temporally biased manner.


An obvious argument against general morphogenic insulation
in my own data is that \tgf 3 appears
to activate the BMP-specific Smads (Smad1/5/8). I note that
I was unable to verify that this crosstalk is specific (i.e.
not due to a contaminant) as addition of an anti-\tgf\
antibody along with \tgf 3 treatment did not significantly block
the interaction (data not shown). For this reason, I only
interpret the BMP/\tgf\ crosstalk data to say that one does not
inhibit the other.


An additional counter-argument is that these pathways
may in fact interact, but I have measured the wrong
thing to be able to identify the interactions. This would
be a useful discovery, as what my work shows is that
the generally-accepted method of encoding used by these
pathways shows complete insulation. It seems to me quite
possible that there exists some information channel, at
the level of signaling, that does indeed integrate information
from both Wnt and \tgf\ signals.


\subsection{BMP and \tgf\ do not compete for Smad4}


While it is widely believed that co-activation of the
BMP and \tgf\ pathways should result in some sort of
cross-pathway inhibition via Smad4 competition, I find
no evidence of this phenomenon. Indeed, I am unaware
of any studies that explicitly show this form of crosstalk,
and some studies have shown that Smad4 is highly
abundant \cite{Nicolas2004,Clarke2006}.


Interestingly, my data show that even when Smad4
is brought down to limiting levels, there is still no
cross-pathway inhibition. Smad4 is seen as
the factor that allows receptor-Smad (rSmad) access to
the nucleus, and it is generally believed that this
interaction is stoichiometric and non-transient \cite{Warmflash2012}. 
If this were the case, how could limitation of Smad4
selectively ablate one arm of Smad responses but not the other?


One possibility is that my observed difference between an
ablated Smad2/3 and maintained pSmad1/5/8 responses
(\ar{fig:insulation:bmpTgfbInsulation}e) is due
to some difference between the behavior of the bulk protein on one hand
versus the phosphorylated form on the other. However, such an effect would
be difficult to reconcile with the standard model that
the phospho-state allows Smads to stay in the nucleus
(i.e. the nuclear phospho- and total-protein levels should vary together).


What is, to me, a more likely explanation is that one of the
consequences of ablating Smad4 signaling, wich requires 48 hours
of transfection, is a change to the transcriptional network
with respect to \tgfbsf\ pathway components. For example, loss
of Smad4 may have resulted in a decrease in the quantity of \tgf\
receptors, co-receptors, or even an increase in secreted
antagonists. I did not test this, and therefore cannot make
a conclusion with respect to the differential effect of Smad4 RNAi
on Smad2/3 and pSmad1/5/8. Therefore, the most reasonable takeaway from this
data is simply that reduction in Smad4 does not force pathway
crosstalk.


The lack of competition for even low levels of Smad4 leads
to a potential modification of the current model of Smad
nucleo-cytoplasmic shuttling (see \ar{pathways:tgfb:smads}).
It may be that interactions of the
rSmads with Smad4 are more transient than is currently
believed, and that association with Smad4 is not required
for nuclear receptor-Smad activity and maintained localization.
In such a case, Smad4 could behave more like an enzyme
than a stoichiometric scaffold, in that single Smad4 molecules
could mediate the nuclear translocation of numerous rSmads.
If this process were fast enough, then the two classes of rSmads
would effectively have access to different pools of Smad4;
activity of one pathway would not ``soak up'' the co-Smad even
at pathway saturation.


\subsection{Future directions}


While this work has demonstrated that morphogenic
pathway integration may not be as complicated as we think,
there are many unanswered questions. Perhaps the most straightforward
questions to address are on the generality of morphogenic
insulation. Does insulation extend to other classic morphogenic pathways
(such as Hedgehog and Notch)? Does it extend to yet more
diverse cell lines than those tested here? I suspect that
signal transduction insulation is a general principle, and future
work using the approaches
in this chapter could provide the answers to these questions. 


It would also be informative to study the kinetics of crosstalk.
For example, in the case of HCECs, which show cross-pathway
modulation of transcription factors after 18 hours
(\ar{fig:insulation:wntTgfbInsulation}),
one could measure how much time is required after the
signal this crosstalk becomes apparent. Such data could be used
to determine how
long it takes for a cell to build a new, biased model of
its environment, and how stable that biased model is in the presence
of either a constant or changing signal.


Throughout this dissertation I have been harping on this
concept of information transfer, as we must always make
assumptions regarding which cellular and protein properties
are carrying signaling information. My data for Wnt and
\tgfbsf\ suggest that these pathways carry relatively little information
about absolute extracellular ligand concentrations. This lack
of information implies either that cells are terrible concentration
detectors or that we are not looking at the right encoding
relationship between these ligands and the internal cellular model of those
ligands. A comprehensive study designed to sort out the sources
of information, and the precise intracellular properties into which
that information is encoded, will be essential to our understanding
of how cells make decisions as a result of \tgfbsf\ and Wnt signals.






