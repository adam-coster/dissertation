\section{Discussion}
\label{pathways:discussion}

The preceding chapter provided a glimpse of the complexity and uncertainty
in the properties of Wnt and \tgfbsf\ signaling and inter-pathway crosstalk.
Here, I summarize the salient points that lead to the approaches and hypotheses
of my experimental work in \ar{insulation:introduction}.


\subsection{Wnt and \tgfbsf\ use morphogenic encoding systems}

The consensus in the literature is that Wnt and \tgfbsf\ signaling are
morphogenic, meaning that they encode information about ligand concentrations.
As I have noted throughout this text, the fact that these pathways
yield concentration-dependent effects does not imply that this is the only
information about the signals that cells are using.


In
\ar{imaging:information} I discuss how the information content
of biological outputs can be quantified using the mutual information metric.
Published reports indicate that the information content of many biological
input/output relationships, when measured by imaging,
is quite low \cite{Cheong2011}. My own
measurements, using the careful image correction discussed in 
\ar{imaging:introduction}, yield only slightly higher information content.
This low mutual information between ligand concentrations and the
outputs of signaling imply that cells can only effectively determine
whether a signal is present or not; they cannot determine a precise
absolute concentration. In effect, these morphogens are not
morphogenic at the single-cell level!
This limited information content of the concentration-based encoding system is
especially interesting in light of the dramatic network
bottlenecking of both pathways. Together, the implication is that
cells can neither determine which of many
different ligands is present in the environment nor the precise concentrations of those
ligands. How much information about these pathways, then, do single
cells have access to?


On the other hand, we may be neglecting additional information channels
that cells use to encode more accurate models of their extracellular
environments. ``Non-canonical'' information channels may provide
just such content. Or, just like the fly embryo syncytium that
makes use of multiple noisy
transcription factor gradients to obtain accurate positional
information (see \ar{introduction:encoding}) \cite{Gregor2007,Dubuis2011},
cellularized systems such as the intestinal crypt may integrate Wnt and \tgfbsf\
concentrations in order to more accurately define fates or positions along
the crypt axis.


\subsection{Signaling crosstalk reduces information content}


In the preceding sections and in my primary experimental work in
\ar{insulation:introduction}, I take care to classify pathway crosstalk
into distinct types: pre-nuclear signaling crosstalk versus nuclear
crosstalk at the level of transcription.
This is for an important reason, which is that
integrating pathways at the level of signal transduction
causes a loss of information content available to the nucleus.
In other words, the nucleus loses
knowledge about the environment when pathways intersect during signaling.


As an example, the cell makes an internal model of environmental \tgfbsf\
using concentrations of active Smad.
If \tgfbsf\ were the only thing that could activate Smad, then the cell
would ``know'' that \tgfbsf\ was present in the environment any time that
active Smad levels increased. But what if Wnt could also modulate Smad?
In this case, when the nucleus sees changes to active Smad levels, it cannot
know whether \tgfbsf, Wnt, or some combination of the two ligands are
present in the environment. This contrasts to the case where \tgfbsf\
ligands only modulate Smad, and Wnt only modulates \bcat, so that the
internal model within the cell models the more complex
reality of the external environment.


\subsection{Informational asymmetry between the cell and its environment}


Perhaps cells only need limited information about their environments,
such that the reduction of information caused by complex inter-pathway
processing during signal transduction is of no consequence. This would
be surprising, however, as these very same cells are what create the
information-rich extracellular milieu in the first place
(e.g. by secreting signals and otherwise modifying the environments of their
neighbors). How could information-poor
intracellular networks create information-rich extracellular environments?


One possibility is that each cell type present in a complex environment
does indeed have limited information about that environment. It could
then be the case that the environmental complexity is
simply due to the combination of low-information outputs from many distinct cell types.
In effect, individual cell types would be insulated from the complexity
of their environments, and would only have to worry about processing the
few signals that they are capable of ``understanding.''


\subsection{Uncertainty in the nodes of \tgfbsf/Wnt signaling crosstalk}


Crosstalk at the level of signaling reduces what a cell can know about
its environment. It is therefore important to determine whether such crosstalk
truly exists before speculating on why a cell would need so little
environmental information, as I so prematurely began to do above.


Cellular signaling is extremely difficult to study, for the reasons
outlined in \ar{introduction:introduction}, and the studies
cited in the current chapter are illustrative of this fact.
While it is possible that the results of all of the cited studies are accurate,
they must be interpreted with care. In particular, all of the cited
studies that identified nodes of \tgfbsf/Wnt crosstalk relied on
some combination of overexpression, RNAi knockdown, or pharmacological
inhibition of pathway components. All of these methods can push the
global cellular signaling network into states that normal cells
cannot inhabit. This is especially true for studies involving
\bcat\ and Axin, as these proteins are scaffolds that are normally present at extremely
low concentrations.


Also, the further apart experimental inputs and outputs
are in time, the less accurate our inferences can be about the mechanisms
connecting them. The cited studies rely heavily on co-immunoprecipitation of
overexpressed pathway components to demonstrate the possibility of
interactions. They then use transcriptional reporters
and other long-term readouts to measure the consequences of these interactions.
However, a relationship between physical interactions
and downstream consequences can only be demonstrated by specifically blocking
that interaction. This has not been done for any node of
putative \tgfbsf/Wnt signaling crosstalk, and for good reason: such an
experiment is exceedingly difficult to design.


In short, current studies on \tgfbsf/Wnt signaling crosstalk do not
show either that such crosstalk definitively exists, nor that the crosstalk
has functional consequences if it does exist.
Indeed, in \ar{insulation:introduction} I show strong evidence under
endogenous signaling conditions that
these pathways are insulated from one another during signaling. This
implies that nuclei maintain more accurate models of their environments by
integrating \tgfbsf\ and Wnt primarily at the level of transcription.


It will be important for future studies to accurately
determine the information content of the \tgfbsf\ and Wnt signaling pathways,
and how that information is encoded by the cell. 
Such studies may reveal that cells encode more about the signal than
just its concentration.
In combination with the results of \ar{insulation:introduction},
that the integration of these
pathways gives cells more information, we may find that
cells create more accurate internal models of their extracellular
environments than is currently believed.

